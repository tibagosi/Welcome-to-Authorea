\chapter{Cognitive Agents}

Cognitive agents, by the definition of (cite) are pieces of software that perceive and act on an environment. Agents play a very important role in our project, namely, they represent each actor of the real world rescue team in a multi-agent system. The management of the high-level information between the high-level database and the visual interface is done by each agent, depending on the actor it represents. Hence, agents 

\begin{itemize}
\item GOAL agent programming framework

The GOAL agent programming framework (cite) was introduced by (cite), and represents today a powerful tool for agent development. It is a Java-based program with its own GOAL language for agent definition and basic program constructs (if rules, for loops, message, percept and goal operators). 
In GOAL, a multi-agent system consists of agents that get percepts from an environment, perform actions on it, and send messages between each other. Each agent is defined by a set bases: a knowledge, a belief, a percept, a message and a goal base. Described more in detail here (cite).
At the current state, it had support for Prolog as a Knowledge Representation Technology for agents. 
As the project requires ontologies to be the way of representing knowledge, agents are now faced with the problem of using ontology technologies for KR purposes. This introduced the need for a change: instead of hard-wiring now ontology technologies into GOAL, a better, more generic approach was taken to be able to plug in from now on any KR technology for the use of agents. As discussed more in detail in the paper (cite), the generic KR Interface was created and used for this purpose.

\item KR Interface

The KR Interface is a generic interface for cognitive agents to be able to use any KRT (under some basic assumptions) for representing the agents' knowledge. 


\item OWL mental state

\item OWL agents

\item Semantic web to agents



\end{itemize}
  
  
  