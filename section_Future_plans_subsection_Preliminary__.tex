\section{Future plans}
\subsection{Preliminary and Expected Results}

The preliminary publication results includes a workshop paper at AAMAS entitled 'Designing a Knowledge Representation Interface for Cognitive Agents' \cite{}, as attached to this document. 
At the Bessensap 'Press meets research' event, I gave a talk about 'Creating effective Human-Robot collaboration with Ontologies'. 
The current status of progress on the main research topics are: 
\begin{itemize}
\item Progress on the \textbf{ontology} is at a phase where we are ready to write a paper about the design of the ontology. As described in Section 5, there are many aspects (ingredients) that we needed to consider, corresponding to the many requirements the ontology needs to fulfill. As such, related literature was studied, during the project exercises we managed to question the firefighters for some domain expert knowledge, and collected enough data and experience about a mission, to determine the high-level information to be captured by the ontology. 
\item At the \textbf{agents} front, creating a first OWL agent necessitated some language redesign of GOAL, as well as refactoring due to the integration of two new interfaces: KR and MST, as described in Section 6. This work already produced a paper on the KR Interface, and will produce another one intended to be published in a journal as well, that would include the MST Interface as well as the language redesign to be included in the description. 
\item Work on the \textbf{display logic} was mostly done for the TRADR project exercise meetings, where a basic display logic included show/no show properties for the newly appearing images of the mission. At this point the agent did not take into account the role and geographical position of the actor, since the project display system was also in a pre-prototypical state, and such informations could not be furnished. 
\end{itemize}

In the near future, another journal paper on completing the agent design and implementation is awaiting submission. A conference paper on the ontology and the display logic is planned after conducting an end-user experiment on how to achieve situational awareness with furnishing or not different types of information to different users.
The future plan on each dimension includes:
\begin{itemize}
\item \textbf{ontology}  To validate and evaluate the constructed and gradually improved ontology against all requirements and users of the project during TRADR evaluation and experimentation events. To conduct an end-user experiment to determine the level of adequacy of the ontology. To evaluate the ontology from the graphical interface's and cognitive agents' point of view.  
The current ontology will go through major changes as all the modules and the comprehensive ontology design is worked out to the detail. A full validation and evaluation is following the pipeline for ensuring the level of adequacy of the ontology to the project.
\item \textbf{agents} In the distant future, more reasoning power is requested from the agents, that now should be able to have a very high control over the visual interface of the user. 
Comparison OWL-Prolog agents
\item \textbf{display logic} End-user experiment and paper about what type of information to show to whom, which categories, levels of priorities, how does it affect CTL.
\end{itemize}

\subsection{Publication Plan}

The publication plan consists of one to two high-ranked conference papers or one journal paper per year. 
\begin{itemize}
\item Year 1 : workshop paper at AAMAS 2015
\item Year 2 : conference paper on the ontology design, journal paper on the agents design, conference paper on the display logic features
\item Year 3 : conference paper on the evaluation of the ontology, conference paper on the comparison of OWL and Prolog agents
\item Year 4 : dissertation 
\end{itemize}

\subsection{Doctoral Education Plan}
The doctoral education plan foresees a total of 45 credit points in the course of the total 4 years of the doctorate program. There are several ways to achieve credits, and these are categorized into three: research-related, transferrable skills and disciplinary blabla. So far, x credits have been achieved in the category of transferrable skills through attending classes organized by the Graduate School, and x credits have been achieved from attending classes on the topic, and 1 credit for an accepted workshop paper at the AAMAS 2015 conference on Autonomous Agents and Mulit-Agent Systems.
In the future, more research skills are planned.
Teaching a lecture tutorial.
Other papers as described in the previous chapters.

  
\subsection{Execution Timeline}  
  
  
  
  