\chapter{The USAR Ontology}
The USAR Ontology is aimed to capture the high-level information of the TRADR project, fulfilling several requirements. First, to represent the information that is being discovered during a mission, such that this can be shown on a visual interface, as well as used by the agents for reasoning. 
The environment is represented as a map with areas, points of interest, hazards such as fire or explosion. The actors of the operation (robots and humans) are all geographically positioned on the map, and hence stored in the ontology as well. An agent stands for an actor, and each actor present has an agent, to map the reality to a multi-agent system. 

Agents operate on this ontology in order to determine the display logic (as described in Chapter 5), the mapping of roles and capabilities to visual elements on the interface. Thus, the ontology serves as the base core to create situational awareness on the team level. That means that every member of the rescue team has the perfect overview of the situation he/she needs, given his/her role, task, capabilities, current status and activity. The ontology hence should capture information about a member, as well as a team, a mission, a sortie, etc.

\end{itemize}
  
  
  
  
  
  
  
  